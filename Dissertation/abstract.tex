\chapter*{РЕФЕРАТ}							% Заголовок

Дипломна робота: \formbytotal{TotPages}{c.}{}{}{}, 
~\formbytotal{totalcount@figure}{рис.}{}{}{},
~\formbytotal{totalcount@table}{табл.}{}{}{},
2 дод.,
\formbytotal{citenum}{джер.}{}{}{}.
ІНТЕЛЕКТУАЛЬНИЙ АНАЛІЗ ДАНИХ, КЛАСИФІКАЦІЯ, КЕРАС,
ОБРОБКА ПРИРОДНОЇ МОВИ, КОНВОЛЮЦІЙНІ НЕЙРОННІ МЕРЕЖІ, ГЛИБИННЕ НАВЧАННЯ.\\
Об’єктом дослідження є оголошення на платформі для електронної комерції. Предметом дослідження є модель для класифікації оголошення.
Мета дослідження:
\begin{enumerate}
	\item розробити необхідний інструментарій для перетворення текстової інформації вектори
	\item дослідження методів та алгоритмів для класифікації текстів та її перевірки;
	\item розробка ПЗ, який реалізує алгоритми класифікації;
    \item аналіз результатів.
\end{enumerate}

Теоретичною та методологічною основою дослідження є праці
закордонних вчених в галузі інтелектуальної обробки даних, математичного
моделювання, класифікації даних та маркетингу. \\
В ході дипломної роботи створено програмний продукт для класифікації оголошень, проведено аналіз результатів роботи програми на реальних даних. \\
Методологія реалізована на основі уже відомих глибинних нейроних мереж: конволюційних та рекурентних з використанням власних розробок, що містять особливу архітектуру та використання регуляризацій для запобігання перенавчанню моделей.\\
Програмний продукт реалізовано за допомогою мови програмування
Python та фреймворку для роботи с нейронними мережами Keras. Надано
рекомендації до подальших досліджень. \\


\chapter*{ABSTRACT}						


Thesis work: \formbytotal{TotPages}{pp.}{}{}{}, 
~\formbytotal{totalcount@figure}{fig.}{}{}{},
~\formbytotal{totalcount@table}{tabl.}{}{}{},
2 app.,
\formbytotal{citenum}{cit.}{}{}{}.
DATA MINING, CLASSIFICATION, KERAS, NATURAL LANGUAGE PROCESSING, CONVOLUTION NEURAL NETWORKS, DEEP LEARNING \\
The object of study is the is advertisements at e-commerce
platform. The subject of study is is classification model for
advertisements.
The aim of the study:

\begin{enumerate}
	\item research methods and algorithms for text transformation into vectors;
	\item investigate algorithms and methods for text classification;
	\item build an software for advertisements classification
	\item analyze of results 
\end{enumerate}

Theoretical and methodological basis of the study is the work of foreign foreign scientists in the field of data mining, mathematical modeling, data classification and marketing. \\
During the thesis created software to classify advertisements using their title and descriptions, and present the results of the program on real data.\\
The methodology is implemented on the basis of already known Deep Neural Networks: convolution and recurrent using my own development, which include special architecture of neural networks and use of special regularizations to overcome overfitting problem. \\
The software is implemented using the Python programming language and
framework for working with Neural Networks Keras. Recommendations for further research are given. \\
\clearpage
